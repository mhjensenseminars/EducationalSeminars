
% LaTeX Beamer file automatically generated from DocOnce
% https://github.com/hplgit/doconce

%-------------------- begin beamer-specific preamble ----------------------

\documentclass{beamer}

\usetheme{red_plain}
\usecolortheme{default}

% turn off the almost invisible, yet disturbing, navigation symbols:
\setbeamertemplate{navigation symbols}{}

% Examples on customization:
%\usecolortheme[named=RawSienna]{structure}
%\usetheme[height=7mm]{Rochester}
%\setbeamerfont{frametitle}{family=\rmfamily,shape=\itshape}
%\setbeamertemplate{items}[ball]
%\setbeamertemplate{blocks}[rounded][shadow=true]
%\useoutertheme{infolines}
%
%\usefonttheme{}
%\useinntertheme{}
%
%\setbeameroption{show notes}
%\setbeameroption{show notes on second screen=right}

% fine for B/W printing:
%\usecolortheme{seahorse}

\usepackage{pgf}
\usepackage{graphicx}
\usepackage{epsfig}
\usepackage{relsize}

\usepackage{fancybox}  % make sure fancybox is loaded before fancyvrb

\usepackage{fancyvrb}
%\usepackage{minted} % requires pygments and latex -shell-escape filename
%\usepackage{anslistings}
%\usepackage{listingsutf8}

\usepackage{amsmath,amssymb,bm}
%\usepackage[latin1]{inputenc}
\usepackage[T1]{fontenc}
\usepackage[utf8]{inputenc}
\usepackage{colortbl}
\usepackage[english]{babel}
\usepackage{tikz}
\usepackage{framed}
% Use some nice templates
\beamertemplatetransparentcovereddynamic

% --- begin table of contents based on sections ---
% Delete this, if you do not want the table of contents to pop up at
% the beginning of each section:
% (Only section headings can enter the table of contents in Beamer
% slides generated from DocOnce source, while subsections are used
% for the title in ordinary slides.)
\AtBeginSection[]
{
  \begin{frame}<beamer>[plain]
  \frametitle{}
  %\frametitle{Outline}
  \tableofcontents[currentsection]
  \end{frame}
}
% --- end table of contents based on sections ---

% If you wish to uncover everything in a step-wise fashion, uncomment
% the following command:

%\beamerdefaultoverlayspecification{<+->}

\newcommand{\shortinlinecomment}[3]{\note{\textbf{#1}: #2}}
\newcommand{\longinlinecomment}[3]{\shortinlinecomment{#1}{#2}{#3}}

\definecolor{linkcolor}{rgb}{0,0,0.4}
\hypersetup{
    colorlinks=true,
    linkcolor=linkcolor,
    urlcolor=linkcolor,
    pdfmenubar=true,
    pdftoolbar=true,
    bookmarksdepth=3
    }
\setlength{\parskip}{0pt}  % {1em}

\newenvironment{doconceexercise}{}{}
\newcounter{doconceexercisecounter}
\newenvironment{doconce:movie}{}{}
\newcounter{doconce:movie:counter}

\newcommand{\subex}[1]{\noindent\textbf{#1}}  % for subexercises: a), b), etc

%-------------------- end beamer-specific preamble ----------------------

% Add user's preamble




% insert custom LaTeX commands...

\raggedbottom
\makeindex

%-------------------- end preamble ----------------------

\begin{document}

% matching end for #ifdef PREAMBLE

\newcommand{\exercisesection}[1]{\subsection*{#1}}



% ------------------- main content ----------------------

% Slides for PHY981


% ----------------- title -------------------------

\title{\href{{http://www.nucleartalent.org}}{Nuclear TALENT}:perspectives and future plans}

% ----------------- author(s) -------------------------

\author{Morten Hjorth-Jensen, National Superconducting Cyclotron Laboratory and Department of Physics and Astronomy, Michigan State University, East Lansing, MI 48824, USA {\&} Department of Physics, University of Oslo, Oslo, Norway\inst{}}
\institute{}
% ----------------- end author(s) -------------------------

\date{Second China-US RIB meeting, Beijing, October 16-18, 2017
% <optional titlepage figure>
% <optional copyright>
}

\begin{frame}[plain,fragile]
\titlepage
\end{frame}

\begin{frame}[plain,fragile]
\frametitle{\href{{http://www.nucleartalent.org}}{What is Talent}}

\begin{block}{}
\textbf{The TALENT initiative aims at providing an advanced and comprehensive training to graduate students and young researchers in low-energy nuclear theory and experiment.}

The network aims at developing a broad curriculum that will provide the platform for a cutting-edge theory for understanding nuclei and nuclear reactions. These objectives are being met by offering intensive three-week courses on a rotating set of topics, commissioned from experienced teachers in nuclear theory and experiment. The educational material generated under this program will be collected in the form of web-based courses, textbooks, and a variety of modern educational resources. No such all-encompassing material is available at present; its development will allow dispersed university groups to profit from the best expertise available.

\end{block}
\end{frame}

\begin{frame}[plain,fragile]
\frametitle{Motivations and aims}

\begin{block}{}
\begin{itemize}
\item Develop structured modules which will provide our students with a modern education in nuclear physics

\item Help smaller university groups which cannot cover the whole range and breadth of nuclear physics

\item Modules/courses should contain a high-level of synchronization

\item A computational perspective is essential

\item plus many other aspects
\end{itemize}

\noindent
\end{block}
\end{frame}

\begin{frame}[plain,fragile]
\frametitle{\href{{http://fribtheoryalliance.org/TALENT/content/board.php}}{Present Talent board}}

\begin{block}{}
\begin{itemize}
\item Northern American Board members:
\begin{itemize}

  \item Dick Furnstahl (Ohio State University), heading the board

  \item Chuck Horowitz (Indiana University) 

  \item Filomena Nunes (Michigan State University) 

  \item Martin Savage (University of  Washington) 

\end{itemize}

\noindent
\item European Board Members:
\begin{itemize}

  \item Jacek Dobaczewski (Poland, Finland, and UK) 

  \item Christian Forssen (Chalmers, Sweden) 

  \item Francesca Gulminelli (Caen and GANIL, France) 

  \item Morten Hjorth-Jensen (MSU, USA and University of Oslo, Norway)

  \item Jeff Tostevin (Surrey, UK) 
\end{itemize}

\noindent
\end{itemize}

\noindent
Board members serve for periods up to six years. With the strong interest in China, the board should be extended to include members from Asian countries
\end{block}
\end{frame}

\begin{frame}[plain,fragile]
\frametitle{\href{{http://fribtheoryalliance.org/TALENT/courses/courses.php}}{Nuclear Talent courses}  offered 2012-2017}

\begin{block}{}
In total we have organized 12 courses at several different places
\begin{enumerate}
\item Nuclear forces (INT 2013)

\item Many-body methods (\textbf{GANIL July 2015})

\item Few-body methods for nuclear physics (\textbf{ECT 2015})

\item Density functional theory and self-consistent methods (\textbf{ECT 2014} and \textbf{York 2016})

\item Theory for exploring nuclear structure experiments (\textbf{GANIL 2014} and \textbf{ECT 2017})

\item Theory for exploring nuclear reaction experiments (\textbf{GANIL 2013})

\item Nuclear theory for astrophysics (\textbf{MSU 2014} and \textbf{INT 2015})

\item Theoretical approaches to describe  exotic nuclei (planned for 2018/2019, INT)

\item High-performance computing and computational tools for nuclear physics (\textbf{ECT 2012} and \textbf{North Carolina State University 2016})
\end{enumerate}

\noindent
\end{block}
\end{frame}

\begin{frame}[plain,fragile]
\frametitle{\href{{http://fribtheoryalliance.org/TALENT/courses/courses.php}}{2017 and plans for 2018/2019 Nuclear Talent courses}}

\begin{block}{}
For all courses (till this year) we have had on average more than 40 applicants per course. The Shell-model course held this year had 52 applicants and we accepted 31 students. They came from 16 different countries and 20 different institutions/laboratories.


\begin{enumerate}
\item \href{{https://github.com/NuclearTalent/NuclearStructure}}{Theory for exploring nuclear structure experiments}, \textbf{ECT 2017}

\item Many-body methods for nuclear structure and reactions (courses 5 and 6), July-August 2018 at \href{{http://www.htu.cn/english/}}{Henan Normal University}, Xin-Xiang with Chung-Wang Ma as local host

\item The first edition of Course 10 on Fundamental Symmetries will be offered at the INT in the summer of 2018. Contact: Michael Ramsey-Musolf (mjrm@physics.umass.edu)

\item The first edition of Course 8 on Theoretical approaches to describe exotic nuclei will be titled Atomic Nuclei as Open Quantum Systems: Unifying Nuclear Structure and Reactions. It will be offered in the summer of 2019 at the \textbf{INT}. Contact: Christian Forssen (christian.forssen@chalmers.se)
\end{enumerate}

\noindent
The second edition of Course 6: Theory for exploring nuclear reaction experiments, will most likely be postponed to later.

\end{block}
\end{frame}

\begin{frame}[plain,fragile]
\frametitle{The 2018 course at \href{{http://www.htu.cn/english/}}{Henan Normal University, Xin-Xiang}.}

\begin{block}{}
The aim is to run a three-week Talent course at Henan Normal University in Xin-Xiang. Chung-Wang Ma is the local host. 

The Chinese community had a course on Reaction theory for nuclear experiments (course 6) as the ideal. It has however been difficult to set up this course and a possibility is to rather organize a course on

\begin{itemize}
\item Many-body methods for nuclear structure and reactions (courses 5 and 6), July-August 2018.
\end{itemize}

\noindent
\end{block}
\end{frame}

\begin{frame}[plain,fragile]
\frametitle{The 2018 course at \href{{http://www.htu.cn/english/}}{Henan Normal University, Xin-Xiang}. Possible paths}

\begin{block}{}
We have already a group of potential organizers and teachers and the following possibilities could be
\begin{enumerate}
\item Focus on two central (and partly new) many-body methods like Coupled Cluster theory and In-medium Renormalization group theory with applications to nuclear structure studies and reaction theory studies.

\item Focus more on shell-model studies and for example coupled-cluster theory with applications to nuclear structure studies and inputs to reaction theory studies
\end{enumerate}

\noindent
\end{block}
\end{frame}

\begin{frame}[plain,fragile]
\frametitle{The 2018 course at \href{{http://www.htu.cn/english/}}{Henan Normal University, Xin-Xiang}, China}

\begin{block}{}
The first path
\begin{itemize}
\item Many-body methods like Coupled Cluster theory and In-medium Renormalization group theory with applications to nuclear structure studies and reaction theory studies
\end{itemize}

\noindent
would favor more theoretically inclined students while
\begin{itemize}
\item Shell-model studies (with perhaps an additional many-body method) as basis for applications to nuclear structure studies and inputs to reaction theory studies
\end{itemize}

\noindent
could be of interest for both theorists and experimentalists since the shell-model approach has a lower level of complexity than the other two many-body methods.
Quantities like spectroscopic factors and one-body densities which enter reaction theory studies are easier to study within a shell-model framework.


\end{block}
\end{frame}

\begin{frame}[plain,fragile]
\frametitle{The 2018 course at \href{{http://www.htu.cn/english/}}{Henan Normal University}, important sides of the Talent courses}

\begin{block}{}
Important aspects of the Talent courses have been
\begin{itemize}
\item students aquire skills and konwledge that would have been difficult or almost impossible to aquire while following traditional courses

\item most Talent courses have computational elements where students write code that have many of the elements of codes used in nuclear physics research.

\item close contact between teachers and students. 
\end{itemize}

\noindent
\end{block}
\end{frame}

\begin{frame}[plain,fragile]
\frametitle{The 2018 course at \href{{http://www.htu.cn/english/}}{Henan Normal University}, questions to think of}

\begin{block}{}

The nuclear shell-model is an example of codes which can be written and developed for simpler systems during a Talent course of three weeks. Combined with a professional research code like NushellX or similar nuclear structure tools, it is easier to understand the input and the theory which the analysis of nuclar structure and reaction experiments. 

The hands-on aspects of the Talent courses is highly appreciated by the participants.  We may need to think of
\begin{itemize}
\item which skills and learning outcomes do we want our students to aquire? See the \href{{https://github.com/NuclearTalent/NuclearStructure}}{2017 course as an example}

\item and what is pedagogically possible to achieve?

\item and what is desirable and useful to the Chinese community? 
\end{itemize}

\noindent
\end{block}
\end{frame}

\begin{frame}[plain,fragile]
\frametitle{Nuclear Talent v2.0, questions}

\begin{block}{}
\begin{itemize}
\item Is it possible to integrate material developed in different Talent courses, offering thereby a coherent source for educating the next generation of nuclear physicists? Keyword: Modularization of topics.

\item How many basic courses can an institution offer, and which courses should be offered?

\item How can we  coordinate an advanced training in nuclear physics?

\item Can we integrate the (\emph{ad hoc}) Nuclear Talent courses/initiative  in our education?

\item Can we use efficiently version control tools  like \href{{https://git-scm.com/}}{git} and \href{{https://github.com/}}{Github} or other modern tools? 
\end{itemize}

\noindent
\end{block}
\end{frame}

\begin{frame}[plain,fragile]
\frametitle{Nuclear Talent v2.0, technicalities: Scientific writing and publishing for the future}

\begin{block}{}
Scientific writing = {\LaTeX}

\begin{enumerate}
\item Pre 1980: handwriting/typewriting + publisher

\item Post 1985: scientists write {\LaTeX}

\item Post 2010: a few scientists explore new digital formats

\item Big late 1990s question: Will MS Word replace {\LaTeX}? It never did!

\item {\LaTeX} PDF is mostly suboptimal for the new devices

\item The book will survive ({\LaTeX} is ideal)

\item The classical report/paper will survive ({\LaTeX} is ideal)

\item But there is an explosion of new platforms for digital learning systems!

\item How to write scientific material that can be easily published through old and new media?
\end{enumerate}

\noindent
\end{block}
\end{frame}

\begin{frame}[plain,fragile]
\frametitle{Can one assemble lots of different writings to a new future document (or even a book)?}

\begin{block}{}
Suppose I write various types of scientific material
\begin{enumerate}
\item {\LaTeX} document,

\item blog posts (HTML),

\item web pages (HTML),

\item Sphinx documents,

\item IPython/Jupyter notebooks,

\item wikis,

\item Markdown files, ...
\end{enumerate}

\noindent
and later want to collect the pieces into a larger document, maybe some book - is that at all feasible?

\end{block}
\end{frame}

\begin{frame}[plain,fragile]
\frametitle{Popular tools anno 2017 and their math support}

\begin{block}{}
\begin{enumerate}
\item {\LaTeX}: de facto standard for math-instensive documents

\item \textsc{pdf}{\LaTeX}, XeLaTeX, LuaLaTeX: takes over (figures in png, pdf) - use these!

\item MS Word: too clicky math support and ugly fonts, but much used

\item HTML with MathJax: "full" {\LaTeX} math, but much tagging

\item Sphinx: somewhat limited {\LaTeX} math support, but great support for web design, and less tagged than HTML

\item reStructuredText: similar to Sphinx, but no math support, transforms to lots of formats ({\LaTeX}, HTML, XML, Word, OpenOffice, ...)

\item Markdown: somewhat limited {\LaTeX} math support, but minor tagging, transforms to lots of formats ({\LaTeX}, HTML, XML, Word, OpenOffice, ...)

\item IPython/Jupyter notebooks: Markdown code/math, combines Python code, interactivity, and visualization, and is becoming a standard in many courses
\end{enumerate}

\noindent
\end{block}
\end{frame}

\begin{frame}[plain,fragile]
\frametitle{More popular tools anno 2017 and their math support}

\begin{block}{}
\begin{enumerate}
\item Confluence: Markdown-like input, with limited {\LaTeX} math support, but converted to XML

\item MediaWiki: quite good {\LaTeX} math support (cf.~Wikipedia/Wikibooks)

\item Other wiki formats: no math support, great for collaborative editing

\item Wordpress: supports full HTML with {\LaTeX} formulas only

\item Google blogger: supports full HTML with MathJax

\item and more...
\end{enumerate}

\noindent
\end{block}
\end{frame}

\begin{frame}[plain,fragile]
\frametitle{DocOnce: one file to rule them all}

\begin{block}{}

\href{{http://hplgit.github.io/doconce/doc/web/index.html}}{DocOnce} offers minimalistic typing, great flexibility wrt format, especially for scientific writing with much math and code. Developed by \href{{http://hplgit.github.io/homepage/index.html}}{Hans Petter Langtangen}, University of Oslo and Simula Research lab

\begin{enumerate}
\item Can generate {\LaTeX}, HTML, Sphinx, Markdown, MediaWiki, Google wiki, Creole wiki, reST, plain text

\item Made for large science books and small notes

\item Targets paper and screen

\item Many special features (code snippets from files, embedded movies, admonitions, modern {\LaTeX} layouts, extended math support for Sphinx/Markdown, ...)

\item Very effective for generating slides from ordinary text

\item Applies Mako: DocOnce text is a program (!)

\item Much like Markdown, less tagged than {\LaTeX}, HTML, Sphinx
\end{enumerate}

\noindent
\end{block}
\end{frame}

\end{document}
