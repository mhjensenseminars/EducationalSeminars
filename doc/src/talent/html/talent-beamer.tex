
% LaTeX Beamer file automatically generated from DocOnce
% https://github.com/hplgit/doconce

%-------------------- begin beamer-specific preamble ----------------------

\documentclass{beamer}

\usetheme{red_plain}
\usecolortheme{default}

% turn off the almost invisible, yet disturbing, navigation symbols:
\setbeamertemplate{navigation symbols}{}

% Examples on customization:
%\usecolortheme[named=RawSienna]{structure}
%\usetheme[height=7mm]{Rochester}
%\setbeamerfont{frametitle}{family=\rmfamily,shape=\itshape}
%\setbeamertemplate{items}[ball]
%\setbeamertemplate{blocks}[rounded][shadow=true]
%\useoutertheme{infolines}
%
%\usefonttheme{}
%\useinntertheme{}
%
%\setbeameroption{show notes}
%\setbeameroption{show notes on second screen=right}

% fine for B/W printing:
%\usecolortheme{seahorse}

\usepackage{pgf,pgfarrows,pgfnodes,pgfautomata,pgfheaps,pgfshade}
\usepackage{graphicx}
\usepackage{epsfig}
\usepackage{relsize}

\usepackage{fancybox}  % make sure fancybox is loaded before fancyvrb

\usepackage{fancyvrb}
%\usepackage{minted} % requires pygments and latex -shell-escape filename
%\usepackage{anslistings}
%\usepackage{listingsutf8}

\usepackage{amsmath,amssymb,bm}
%\usepackage[latin1]{inputenc}
\usepackage[T1]{fontenc}
\usepackage[utf8]{inputenc}
\usepackage{colortbl}
\usepackage[english]{babel}
\usepackage{tikz}
\usepackage{framed}
% Use some nice templates
\beamertemplatetransparentcovereddynamic

% --- begin table of contents based on sections ---
% Delete this, if you do not want the table of contents to pop up at
% the beginning of each section:
% (Only section headings can enter the table of contents in Beamer
% slides generated from DocOnce source, while subsections are used
% for the title in ordinary slides.)
\AtBeginSection[]
{
  \begin{frame}<beamer>[plain]
  \frametitle{}
  %\frametitle{Outline}
  \tableofcontents[currentsection]
  \end{frame}
}
% --- end table of contents based on sections ---

% If you wish to uncover everything in a step-wise fashion, uncomment
% the following command:

%\beamerdefaultoverlayspecification{<+->}

\newcommand{\shortinlinecomment}[3]{\note{\textbf{#1}: #2}}
\newcommand{\longinlinecomment}[3]{\shortinlinecomment{#1}{#2}{#3}}

\definecolor{linkcolor}{rgb}{0,0,0.4}
\hypersetup{
    colorlinks=true,
    linkcolor=linkcolor,
    urlcolor=linkcolor,
    pdfmenubar=true,
    pdftoolbar=true,
    bookmarksdepth=3
    }
\setlength{\parskip}{0pt}  % {1em}

\newenvironment{doconceexercise}{}{}
\newcounter{doconceexercisecounter}
\newenvironment{doconce:movie}{}{}
\newcounter{doconce:movie:counter}

\newcommand{\subex}[1]{\noindent\textbf{#1}}  % for subexercises: a), b), etc

%-------------------- end beamer-specific preamble ----------------------

% Add user's preamble




% insert custom LaTeX commands...

\raggedbottom
\makeindex

%-------------------- end preamble ----------------------

\begin{document}

% endif for #ifdef PREAMBLE



% ------------------- main content ----------------------

% Slides for PHY981


% ----------------- title -------------------------

\title{\href{{http://www.nucleartalent.org}}{Nuclear TALENT}, perspectives and future plans}

% ----------------- author(s) -------------------------

\author{Morten Hjorth-Jensen, National Superconducting Cyclotron Laboratory and Department of Physics and Astronomy, Michigan State University, East Lansing, MI 48824, USA {\&} Department of Physics, University of Oslo, Oslo, Norway\inst{}}
\institute{}
% ----------------- end author(s) -------------------------

\date{Nuclei workshop, MSU and FRIB, June 10-13 2015
% <optional titlepage figure>
}

\begin{frame}[plain,fragile]
\titlepage
\end{frame}

\begin{frame}[plain,fragile]
\frametitle{Motivation}

\begin{block}{}
\begin{itemize}
\item Develop structured modules which will provide our students with a modern education in nuclear physics

\item Modules/courses should contain a high-level of synchronization

\item A computational perspective is essential

\item The FRIB theory center can function as the national coordinating unit
\end{itemize}

\noindent
All material at my github address, go to \href{{https://github.com/mhjensen}}{\nolinkurl{https://github.com/mhjensen}}, click on the seminars link.
\end{block}
\end{frame}

\begin{frame}[plain,fragile]
\frametitle{Nuclear Talent v2.0}

\begin{block}{}
\begin{itemize}
\item Is it possible to integrate material developed in different Talent courses, offering thereby a coherent source for educating the next generation of nuclear physicists? Keyword: Modularization of topics.

\item How many basic courses can an institution offer, and which courses should be offered?

\item How can the coming FRIB theory center be used to coordinate an advanced training in nuclear physics?

\item Can we integrate the (\emph{ad hoc}) Nuclear Talent courses/initiative  in our education? 

\item \textbf{Nuclear Talent initiative and Asia} . The Chinese community is very enthusiastic about the Talent initiative
\end{itemize}

\noindent
\end{block}
\end{frame}

\begin{frame}[plain,fragile]
\frametitle{Local situation at MSU}

\begin{block}{}

We have at MSU a  
\begin{itemize}
\item \href{{https://people.nscl.msu.edu/~witek/Classes/PHY802/NuclPhys802-2015.html}}{basic survey course PHY802}  and three basic nuclear physics courses 

\item \href{{http://nuclearstructure.github.io/PHY981/doc/web/course.html}}{structure}, 

\item \href{{https://people.nscl.msu.edu/~nunes/phy982/phy982web2015.htm}}{reactions and dynamics}  and 

\item \textbf{Nuclear Astrophysics}. 
\end{itemize}

\noindent
These three basic courses have a duration each  of 30-40 hours (2-3 credits).   

They can be taught as a regular one-semester course or half-semester course. There are also experimental courses.



\end{block}
\end{frame}

\begin{frame}[plain,fragile]
\frametitle{Advanced  modules, Nuclear Talent}

\begin{block}{}
\begin{enumerate}
\item Nuclear forces (INT 2013, new version 2017)

\item Many-body methods (\textbf{GANIL July 2015})

\item Few-body methods for nuclear physics (\textbf{ECT* July-August 2015})

\item Density functional theory and self-consistent methods (ECT* 2014 and \textbf{York 2016})

\item Theory for exploring nuclear structure experiments (GANIL 2014)

\item Theory for exploring nuclear reaction experiments (GANIL 2013)

\item Nuclear theory for astrophysics (MSU 2014 and \textbf{INT 2015})

\item Theoretical approaches to describe  exotic nuclei (planned for 2016, Chalmers, Gothenburg)

\item High-performance computing and computational tools for nuclear physics
\begin{itemize}

  \item ECT* 2012, Shell model and variational Monte Carlo

  \item LANL/ORNL in 2016, Monte Carlo methods 
\end{itemize}

\noindent
\end{enumerate}

\noindent
For all courses (till this year) we have had on average 40 applicants per course.

\end{block}
\end{frame}

\begin{frame}[plain,fragile]
\frametitle{Talent v2.0: Scientific writing and publishing for the future}

\begin{block}{}
Scientific writing = {\LaTeX}

\begin{enumerate}
\item Pre 1980: handwriting/typewriting + publisher

\item Post 1985: scientists write {\LaTeX}

\item Post 2010: a few scientists explore new digital formats

\item Big late 1990s question: Will MS Word replace {\LaTeX}? It never did!

\item {\LaTeX} PDF is mostly suboptimal for the new devices

\item The book will survive ({\LaTeX} is ideal)

\item The classical report/paper will survive ({\LaTeX} is ideal)

\item But there is an explosion of new platforms for digital learning systems!

\item How to write scientific material that can be easily published through old and new media?
\end{enumerate}

\noindent
\end{block}
\end{frame}

\begin{frame}[plain,fragile]
\frametitle{Can I assemble lots of different writings to a new future document (book)?}

\begin{block}{}
Suppose I write various types of scientific material
\begin{enumerate}
\item {\LaTeX} document,

\item blog posts (HTML),

\item web pages (HTML),

\item Sphinx documents,

\item IPython notebooks,

\item wikis,

\item Markdown files, ...
\end{enumerate}

\noindent
and later want to collect the pieces into a larger document, maybe some book - is that at all feasible?

\end{block}
\end{frame}

\begin{frame}[plain,fragile]
\frametitle{Popular tools anno 2014 and their math support}

\begin{block}{}
\begin{enumerate}
\item {\LaTeX}: de facto standard for math-instensive documents

\item \textsc{pdf}{\LaTeX}, XeLaTeX, LuaLaTeX: takes over (figures in png, pdf) - use these!

\item MS Word: too clicky math support and ugly fonts, but much used

\item HTML with MathJax: "full" {\LaTeX} math, but much tagging

\item Sphinx: somewhat limited {\LaTeX} math support, but great support for web design, and less tagged than HTML

\item reStructuredText: similar to Sphinx, but no math support, transforms to lots of formats ({\LaTeX}, HTML, XML, Word, OpenOffice, ...)

\item Markdown: somewhat limited {\LaTeX} math support, but minor tagging, transforms to lots of formats ({\LaTeX}, HTML, XML, Word, OpenOffice, ...)

\item IPython notebooks: Markdown code/math, combines Python code, interactivity, and visualization, but requires all code snippets to sync together

\item Confluence: Markdown-like input, with limited {\LaTeX} math support, but converted to XML

\item MediaWiki: quite good {\LaTeX} math support (cf.~Wikipedia/Wikibooks)

\item Other wiki formats: no math support, great for collaborative editing

\item Wordpress: supports full HTML with {\LaTeX} formulas only

\item Google blogger: supports full HTML with MathJax
\end{enumerate}

\noindent
\end{block}
\end{frame}

\begin{frame}[plain,fragile]
\frametitle{DocOnce: one file to rule them all}

\begin{block}{}

\href{{http://hplgit.github.io/doconce/doc/web/index.html}}{DocOnce} offers minimalistic typing, great flexibility wrt format, especially for scientific writing with much math and code. Developed by \href{{http://hplgit.github.io/homepage/index.html}}{Hans Petter Langtangen}, University of Oslo and Simula Research lab

\begin{enumerate}
\item Can generate {\LaTeX}, HTML, Sphinx, Markdown, MediaWiki, Google wiki, Creole wiki, reST, plain text

\item Made for large science books and small notes

\item Targets paper and screen

\item Many special features (code snippets from files, embedded movies, admonitions, modern {\LaTeX} layouts, extended math support for Sphinx/Markdown, ...)

\item Very effective for generating slides from ordinary text

\item Applies Mako: DocOnce text is a program (!)

\item Much like Markdown, less tagged than {\LaTeX}, HTML, Sphinx
\end{enumerate}

\noindent
\end{block}
\end{frame}

\end{document}
